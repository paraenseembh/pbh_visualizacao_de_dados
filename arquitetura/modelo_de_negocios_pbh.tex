\documentclass[12pt,a4paper]{article}
\usepackage[utf8]{inputenc}
\usepackage[brazilian]{babel}
\usepackage{enumitem}
\usepackage{longtable}
\usepackage{geometry}
\usepackage{xcolor}
\usepackage{listings}
\geometry{margin=2cm}

\title{Modelagem de Orçamentos Temáticos com Índices Padronizados:\\Sistema de Classificação 20-50-85-100}
\author{Arquitetura de Dados - Orçamento Público}
\date{\today}

\begin{document}

\maketitle

\tableofcontents
\newpage

\section{Sistema de Índices Padronizados}

\subsection{Conceito Fundamental}

A classificação de dotações orçamentárias em orçamentos temáticos utiliza \textbf{índices discretos padronizados} ao invés de percentuais livres. Os índices válidos são:

\begin{center}
\begin{tabular}{|c|l|p{8cm}|}
\hline
\textbf{Índice} & \textbf{Classificação} & \textbf{Significado} \\
\hline
20 & Tangencial & Relação periférica ou indireta com o orçamento temático \\
\hline
50 & Moderada & Relação significativa, mas não predominante \\
\hline
85 & Predominante & Relação forte e prioritária com o orçamento temático \\
\hline
100 & Exclusiva & Dotação totalmente dedicada ao orçamento temático \\
\hline
\end{tabular}
\end{center}

\subsection{Exemplos Práticos}

\subsubsection{Exemplo 1: Construção de Centro Comunitário}

\textbf{Dotação}: R\$ 1.000.000,00

\textbf{Classificação}:
\begin{itemize}
    \item \textbf{OCA (50)}: 50\% das atividades serão para crianças e adolescentes
    \item \textbf{IDOSO (20)}: 20\% contempla atividades para terceira idade
    \item \textbf{PCD (20)}: 20\% refere-se a acessibilidade e inclusão
\end{itemize}

\textbf{Valores alocados}:
\begin{itemize}
    \item OCA: R\$ 500.000,00
    \item IDOSO: R\$ 200.000,00
    \item PCD: R\$ 200.000,00
    \item \textbf{Não alocado}: R\$ 100.000,00 (10\% - infraestrutura geral)
\end{itemize}

\subsubsection{Exemplo 2: Programa de Educação Infantil}

\textbf{Dotação}: R\$ 800.000,00

\textbf{Classificação}:
\begin{itemize}
    \item \textbf{OCA (100)}: Programa exclusivamente voltado para crianças
\end{itemize}

\textbf{Valores alocados}:
\begin{itemize}
    \item OCA: R\$ 800.000,00
\end{itemize}

\subsubsection{Exemplo 3: Hospital Municipal}

\textbf{Dotação}: R\$ 5.000.000,00

\textbf{Classificação}:
\begin{itemize}
    \item \textbf{OCA (20)}: Atendimento pediátrico
    \item \textbf{IDOSO (20)}: Atendimento geriatria
    \item \textbf{MULHER (20)}: Saúde da mulher
    \item \textbf{PCD (20)}: Reabilitação
\end{itemize}

\textbf{Valores alocados}:
\begin{itemize}
    \item OCA: R\$ 1.000.000,00
    \item IDOSO: R\$ 1.000.000,00
    \item MULHER: R\$ 1.000.000,00
    \item PCD: R\$ 1.000.000,00
    \item \textbf{Não alocado}: R\$ 1.000.000,00 (20\% - clínica geral)
\end{itemize}

\section{Estrutura de Dados Atualizada}

\subsection{Tabela de Domínio: D\_INDICE\_ALOCACAO}

Esta tabela define os índices padronizados aceitos pelo sistema:

\begin{lstlisting}[language=SQL, basicstyle=\small\ttfamily]
CREATE TABLE D_INDICE_ALOCACAO (
    idD_INDICE_ALOCACAO INT PRIMARY KEY,
    VALOR_INDICE INT NOT NULL UNIQUE,
    NOME_CLASSIFICACAO VARCHAR(50) NOT NULL,
    DESCRICAO TEXT,
    ORDEM_PRIORIDADE INT,
    STATUS VARCHAR(20) DEFAULT 'ATIVO',
    
    CHECK (VALOR_INDICE IN (20, 50, 85, 100))
);
\end{lstlisting}

\textbf{Dados iniciais}:
\begin{lstlisting}[language=SQL, basicstyle=\small\ttfamily]
INSERT INTO D_INDICE_ALOCACAO VALUES
(1, 20, 'Tangencial', 
 'Relacao periferica ou indireta com o orcamento tematico. ' ||
 'Indica que a dotacao contempla o publico-alvo de forma ' ||
 'minoritaria ou secundaria.', 
 1, 'ATIVO'),

(2, 50, 'Moderada', 
 'Relacao significativa, mas nao predominante. ' ||
 'A dotacao beneficia o publico-alvo de forma substancial, ' ||
 'mas divide recursos com outros grupos.', 
 2, 'ATIVO'),

(3, 85, 'Predominante', 
 'Relacao forte e prioritaria com o orcamento tematico. ' ||
 'A maior parte dos recursos e atividades sao direcionados ' ||
 'ao publico-alvo especifico.', 
 3, 'ATIVO'),

(4, 100, 'Exclusiva', 
 'Dotacao totalmente dedicada ao orcamento tematico. ' ||
 'Todos os recursos beneficiam exclusivamente o publico-alvo.', 
 4, 'ATIVO');
\end{lstlisting}

\subsection{Dimensão: D\_ORCAMENTO\_TEMATICO}

Sem alterações significativas:

\begin{lstlisting}[language=SQL, basicstyle=\small\ttfamily]
CREATE TABLE D_ORCAMENTO_TEMATICO (
    idD_ORCAMENTO_TEMATICO INT PRIMARY KEY,
    CODIGO_ORCAMENTO VARCHAR(10) NOT NULL UNIQUE,
    NOME_ORCAMENTO VARCHAR(100) NOT NULL,
    SIGLA VARCHAR(10) NOT NULL,
    DESCRICAO TEXT,
    BASE_LEGAL VARCHAR(200),
    PUBLICO_ALVO VARCHAR(100),
    ANO_IMPLANTACAO INT,
    STATUS_ORCAMENTO VARCHAR(20) DEFAULT 'ATIVO',
    EXERCICIO VARCHAR(4)
);
\end{lstlisting}

\end{document}