\documentclass[12pt,a4paper]{article}
\usepackage[utf8]{inputenc}
\usepackage[brazilian]{babel}
\usepackage{geometry}
\usepackage{listings}
\usepackage{xcolor}
\geometry{margin=2cm}

\lstset{
    language=SQL,
    basicstyle=\small\ttfamily,
    keywordstyle=\color{blue}\bfseries,
    stringstyle=\color{red},
    commentstyle=\color{green!50!black},
    showstringspaces=false,
    breaklines=true
}

\title{DDL - Data Warehouse Orçamentário\\Scripts de Criação de Tabelas}
\author{Sistema de Classificação Temática com Índices Padronizados}
\date{\today}

\begin{document}

\maketitle

\tableofcontents
\newpage

\section{Tabelas de Domínio}

\subsection{D\_INDICE\_ALOCACAO}

\begin{lstlisting}
-- =====================================================
-- Tabela: D_INDICE_ALOCACAO
-- Indices padronizados: 20, 50, 85, 100
-- =====================================================
CREATE TABLE D_INDICE_ALOCACAO (
    idD_INDICE_ALOCACAO INT PRIMARY KEY AUTO_INCREMENT,
    VALOR_INDICE INT NOT NULL UNIQUE,
    NOME_CLASSIFICACAO VARCHAR(50) NOT NULL,
    DESCRICAO TEXT,
    ORDEM_PRIORIDADE INT,
    STATUS VARCHAR(20) DEFAULT 'ATIVO',
    DATA_CADASTRO DATETIME DEFAULT CURRENT_TIMESTAMP,
    
    CONSTRAINT chk_valor_indice 
        CHECK (VALOR_INDICE IN (20, 50, 85, 100)),
    CONSTRAINT chk_status_indice 
        CHECK (STATUS IN ('ATIVO', 'INATIVO'))
);

CREATE INDEX idx_indice_valor 
    ON D_INDICE_ALOCACAO(VALOR_INDICE);
CREATE INDEX idx_indice_status 
    ON D_INDICE_ALOCACAO(STATUS);
\end{lstlisting}

\subsection{D\_ORCAMENTO\_TEMATICO}

\begin{lstlisting}
-- =====================================================
-- Tabela: D_ORCAMENTO_TEMATICO
-- Cadastro de orcamentos tematicos (OCA, PCD, IDOSO, etc)
-- =====================================================
CREATE TABLE D_ORCAMENTO_TEMATICO (
    idD_ORCAMENTO_TEMATICO INT PRIMARY KEY AUTO_INCREMENT,
    CODIGO_ORCAMENTO VARCHAR(10) NOT NULL UNIQUE,
    NOME_ORCAMENTO VARCHAR(100) NOT NULL,
    SIGLA VARCHAR(10) NOT NULL UNIQUE,
    DESCRICAO TEXT,
    BASE_LEGAL VARCHAR(200),
    PUBLICO_ALVO VARCHAR(100),
    ANO_IMPLANTACAO INT,
    STATUS_ORCAMENTO VARCHAR(20) DEFAULT 'ATIVO',
    EXERCICIO VARCHAR(4),
    OBSERVACOES TEXT,
    DATA_CADASTRO DATETIME DEFAULT CURRENT_TIMESTAMP,
    DATA_ALTERACAO DATETIME,
    
    CONSTRAINT chk_status_orcamento 
        CHECK (STATUS_ORCAMENTO IN ('ATIVO', 'INATIVO', 'SUSPENSO')),
    CONSTRAINT chk_exercicio_orcamento 
        CHECK (LENGTH(EXERCICIO) = 4)
);

CREATE INDEX idx_orcamento_codigo 
    ON D_ORCAMENTO_TEMATICO(CODIGO_ORCAMENTO);
CREATE INDEX idx_orcamento_sigla 
    ON D_ORCAMENTO_TEMATICO(SIGLA);
CREATE INDEX idx_orcamento_status 
    ON D_ORCAMENTO_TEMATICO(STATUS_ORCAMENTO);
CREATE INDEX idx_orcamento_exercicio 
    ON D_ORCAMENTO_TEMATICO(EXERCICIO);
\end{lstlisting}

\section{Tabelas Dimensão - Estrutura Orçamentária}

\subsection{D\_TEMPO}

\begin{lstlisting}
-- =====================================================
-- Tabela: D_TEMPO
-- Dimensao temporal (exercicios fiscais)
-- =====================================================
CREATE TABLE D_TEMPO (
    idT_TEMPO INT PRIMARY KEY AUTO_INCREMENT,
    EXERCICIO VARCHAR(4) NOT NULL UNIQUE,
    ANO INT,
    STATUS VARCHAR(20) DEFAULT 'ATIVO',
    DATA_CADASTRO DATETIME DEFAULT CURRENT_TIMESTAMP,
    
    CONSTRAINT chk_exercicio_tempo 
        CHECK (LENGTH(EXERCICIO) = 4),
    CONSTRAINT chk_status_tempo 
        CHECK (STATUS IN ('ATIVO', 'ENCERRADO', 'PLANEJAMENTO'))
);

CREATE INDEX idx_tempo_exercicio 
    ON D_TEMPO(EXERCICIO);
CREATE INDEX idx_tempo_ano 
    ON D_TEMPO(ANO);
\end{lstlisting}

\subsection{D\_FUNCAO}

\begin{lstlisting}
-- =====================================================
-- Tabela: D_FUNCAO
-- Classificacao funcional (Portaria MOG 42/1999)
-- =====================================================
CREATE TABLE D_FUNCAO (
    idD_FUNCAO INT PRIMARY KEY AUTO_INCREMENT,
    CODIGO_FUNCAO VARCHAR(4) NOT NULL UNIQUE,
    NOME_FUNCAO VARCHAR(100) NOT NULL,
    DATA_CADASTRO DATETIME DEFAULT CURRENT_TIMESTAMP,
    
    CONSTRAINT chk_codigo_funcao 
        CHECK (LENGTH(CODIGO_FUNCAO) <= 4)
);

CREATE INDEX idx_funcao_codigo 
    ON D_FUNCAO(CODIGO_FUNCAO);
\end{lstlisting}

\subsection{D\_SUBFUNCAO}

\begin{lstlisting}
-- =====================================================
-- Tabela: D_SUBFUNCAO
-- Detalhamento da classificacao funcional
-- =====================================================
CREATE TABLE D_SUBFUNCAO (
    idD_SUBFUNCAO INT PRIMARY KEY AUTO_INCREMENT,
    CODIGO_SUBFUNCAO VARCHAR(4) NOT NULL UNIQUE,
    NOME_SUBFUNCAO VARCHAR(100) NOT NULL,
    DATA_CADASTRO DATETIME DEFAULT CURRENT_TIMESTAMP,
    
    CONSTRAINT chk_codigo_subfuncao 
        CHECK (LENGTH(CODIGO_SUBFUNCAO) <= 4)
);

CREATE INDEX idx_subfuncao_codigo 
    ON D_SUBFUNCAO(CODIGO_SUBFUNCAO);
\end{lstlisting}

\subsection{D\_EIXO}

\begin{lstlisting}
-- =====================================================
-- Tabela: D_EIXO
-- Eixos estrategicos do planejamento
-- =====================================================
CREATE TABLE D_EIXO (
    idD_EIXO INT PRIMARY KEY AUTO_INCREMENT,
    CODIGO_EIXO VARCHAR(10) NOT NULL UNIQUE,
    NOME_EIXO VARCHAR(100) NOT NULL,
    DESCRICAO TEXT,
    DATA_CADASTRO DATETIME DEFAULT CURRENT_TIMESTAMP
);

CREATE INDEX idx_eixo_codigo 
    ON D_EIXO(CODIGO_EIXO);
\end{lstlisting}

\subsection{D\_SUBEIXO}

\begin{lstlisting}
-- =====================================================
-- Tabela: D_SUBEIXO
-- Detalhamento dos eixos estrategicos
-- =====================================================
CREATE TABLE D_SUBEIXO (
    idD_SUBEIXO INT PRIMARY KEY AUTO_INCREMENT,
    CODIGO_SUBEIXO VARCHAR(10) NOT NULL UNIQUE,
    NOME_SUBEIXO VARCHAR(100) NOT NULL,
    CODIGO_EIXO VARCHAR(10),
    DATA_CADASTRO DATETIME DEFAULT CURRENT_TIMESTAMP
);

CREATE INDEX idx_subeixo_codigo 
    ON D_SUBEIXO(CODIGO_SUBEIXO);
CREATE INDEX idx_subeixo_eixo 
    ON D_SUBEIXO(CODIGO_EIXO);
\end{lstlisting}

\subsection{D\_ACAO}

\begin{lstlisting}
-- =====================================================
-- Tabela: D_ACAO
-- Acoes orcamentarias
-- =====================================================
CREATE TABLE D_ACAO (
    idD_ACAO INT PRIMARY KEY AUTO_INCREMENT,
    CODIGO_ACAO VARCHAR(10) NOT NULL UNIQUE,
    NOME_ACAO VARCHAR(200) NOT NULL,
    TIPO_ACAO VARCHAR(50),
    DATA_CADASTRO DATETIME DEFAULT CURRENT_TIMESTAMP
);

CREATE INDEX idx_acao_codigo 
    ON D_ACAO(CODIGO_ACAO);
\end{lstlisting}

\subsection{D\_SUBACAO}

\begin{lstlisting}
-- =====================================================
-- Tabela: D_SUBACAO
-- Detalhamento das acoes orcamentarias
-- =====================================================
CREATE TABLE D_SUBACAO (
    idD_SUBACAO INT PRIMARY KEY AUTO_INCREMENT,
    CODIGO_SUBACAO VARCHAR(45) NOT NULL UNIQUE,
    NOME_SUBACAO VARCHAR(200) NOT NULL,
    CODIGO_ACAO VARCHAR(10),
    METAFISICA_2022 INT,
    METAFISICA_2023 INT,
    METAFISICA_2024 INT,
    METAFISICA_2025 INT,
    PRODUTO VARCHAR(100),
    UNIDADEMEDIDA VARCHAR(50),
    OBJETIVO_ORCAMENTARIO_ID INT,
    VALOR_EMPENHADO DECIMAL(15,2),
    VALOR_LIQUIDADO DECIMAL(15,2),
    VALOR_PAGO DECIMAL(15,2),
    DATA_CADASTRO DATETIME DEFAULT CURRENT_TIMESTAMP
);

CREATE INDEX idx_subacao_codigo 
    ON D_SUBACAO(CODIGO_SUBACAO);
CREATE INDEX idx_subacao_acao 
    ON D_SUBACAO(CODIGO_ACAO);
\end{lstlisting}

\subsection{D\_UO (Unidade Orçamentária)}

\begin{lstlisting}
-- =====================================================
-- Tabela: D_UO
-- Unidades orcamentarias/organizacionais
-- =====================================================
CREATE TABLE D_UO (
    idD_UO INT PRIMARY KEY AUTO_INCREMENT,
    ORGAO_ENTIDADE VARCHAR(100),
    UNIDADE_ORGANIZACIONAL VARCHAR(100),
    SIGLA_UNIDADE_ORGANIZACIONAL VARCHAR(45) NOT NULL UNIQUE,
    UNIDADE_ORGANIZACIONAL_SUPERIOR VARCHAR(100),
    NOME_DO_TITULAR VARCHAR(100),
    ENDERECO VARCHAR(200),
    EMAIL_UNIDADE VARCHAR(100),
    TELEFONE VARCHAR(20),
    DATA_CADASTRO DATETIME DEFAULT CURRENT_TIMESTAMP,
    
    CONSTRAINT chk_email_uo 
        CHECK (EMAIL_UNIDADE LIKE '%@%')
);

CREATE INDEX idx_uo_sigla 
    ON D_UO(SIGLA_UNIDADE_ORGANIZACIONAL);
CREATE INDEX idx_uo_orgao 
    ON D_UO(ORGAO_ENTIDADE);
\end{lstlisting}

\subsection{D\_PROGRAMAS\_PPAG\_ANALITICO}

\begin{lstlisting}
-- =====================================================
-- Tabela: D_PROGRAMAS_PPAG_ANALITICO
-- Programas do Plano Plurianual detalhados
-- =====================================================
CREATE TABLE D_PROGRAMAS_PPAG_ANALITICO (
    idD_PROGRAMAS_PPAG_ANALITICO INT PRIMARY KEY AUTO_INCREMENT,
    EXERCICIO VARCHAR(4) NOT NULL,
    CODIGO_AREA_DE_RESULTADO VARCHAR(45) UNIQUE,
    NOME_AREA_DE_RESULTADO VARCHAR(200),
    CODIGO_DO_PROGRAMA VARCHAR(45) NOT NULL UNIQUE,
    NOME_DO_PROGRAMA VARCHAR(200) NOT NULL,
    OBJETIVO_DO_PROGRAMA TEXT,
    PUBLICO_ALVO_DO_PROGRAMA VARCHAR(200),
    JUSTIFICATIVA_DO_PROGRAMA TEXT,
    NATUREZA_DO_PROGRAMA VARCHAR(100),
    TIPOLOGIA_DO_PROGRAMA VARCHAR(100),
    CLASSIFICACAO_DE_GOVERNO VARCHAR(100),
    DATAINICIO_PROGRAMA DATE,
    DATAFIM_PROGRAMA DATE,
    UNIDADE_GESTORA VARCHAR(100),
    CODIGO_DO_INDICADOR VARCHAR(45),
    NOME_DO_INDICADOR VARCHAR(200),
    FORMA_DE_CALCULO VARCHAR(200),
    UNIDADE_DE_MEDIDA VARCHAR(50),
    FONTE_DE_DADOS VARCHAR(200),
    VALOR_DE_REFERENCIA VARCHAR(45),
    DATA_APURACAO_VALOR_DE_REFERENCIA DATE,
    INDICE_ESPERADO_ANO_1 VARCHAR(45),
    INDICE_ESPERADO_ANO_2 VARCHAR(45),
    INDICE_ESPERADO_ANO_3 VARCHAR(45),
    INDICE_ESPERADO_ANO_4 VARCHAR(45),
    OBSERVACAO TEXT,
    DATA_CADASTRO DATETIME DEFAULT CURRENT_TIMESTAMP,
    
    CONSTRAINT chk_exercicio_ppag 
        CHECK (LENGTH(EXERCICIO) = 4)
);

CREATE UNIQUE INDEX idx_ppag_codigo_programa 
    ON D_PROGRAMAS_PPAG_ANALITICO(CODIGO_DO_PROGRAMA);
CREATE UNIQUE INDEX idx_ppag_codigo_area 
    ON D_PROGRAMAS_PPAG_ANALITICO(CODIGO_AREA_DE_RESULTADO);
CREATE INDEX idx_ppag_exercicio 
    ON D_PROGRAMAS_PPAG_ANALITICO(EXERCICIO);
CREATE INDEX idx_ppag_unidade_gestora 
    ON D_PROGRAMAS_PPAG_ANALITICO(UNIDADE_GESTORA);
\end{lstlisting}

\section{Tabela Fato Central}

\subsection{F\_DOTACAO\_ORCAMENTARIA}

\begin{lstlisting}
-- =====================================================
-- Tabela: F_DOTACAO_ORCAMENTARIA
-- Tabela fato central - dotacoes orcamentarias
-- =====================================================
CREATE TABLE F_DOTACAO_ORCAMENTARIA (
    idF_DOTACAO_ORCAMENTARIA INT PRIMARY KEY AUTO_INCREMENT,
    
    -- Chaves Estrangeiras (Dimensoes)
    idD_PROGRAMAS_PPAG INT,
    idD_UO INT,
    idD_FUNCAO INT,
    idD_SUBFUNCAO INT,
    idD_ACAO INT,
    idD_SUBACAO INT,
    idD_EIXO INT,
    idD_SUBEIXO INT,
    idT_TEMPO INT,
    
    -- Identificadores
    CODIGO_DOTACAO VARCHAR(50) NOT NULL,
    NOME_DOTACAO VARCHAR(200),
    EXERCICIO VARCHAR(4) NOT NULL,
    
    -- Metricas
    VALOR_DOTACAO DECIMAL(15,2) NOT NULL,
    VALOR_EMPENHADO DECIMAL(15,2),
    VALOR_LIQUIDADO DECIMAL(15,2),
    VALOR_PAGO DECIMAL(15,2),
    
    -- Classificacoes adicionais
    NATUREZA_DESPESA VARCHAR(20),
    FONTE_RECURSO VARCHAR(20),
    MODALIDADE_APLICACAO VARCHAR(20),
    ELEMENTO_DESPESA VARCHAR(20),
    
    -- Controle
    STATUS VARCHAR(20) DEFAULT 'ATIVA',
    DATA_CADASTRO DATETIME DEFAULT CURRENT_TIMESTAMP,
    DATA_ALTERACAO DATETIME,
    
    -- Constraints
    CONSTRAINT fk_fdot_programas 
        FOREIGN KEY (idD_PROGRAMAS_PPAG) 
        REFERENCES D_PROGRAMAS_PPAG_ANALITICO(idD_PROGRAMAS_PPAG_ANALITICO),
    CONSTRAINT fk_fdot_uo 
        FOREIGN KEY (idD_UO) 
        REFERENCES D_UO(idD_UO),
    CONSTRAINT fk_fdot_funcao 
        FOREIGN KEY (idD_FUNCAO) 
        REFERENCES D_FUNCAO(idD_FUNCAO),
    CONSTRAINT fk_fdot_subfuncao 
        FOREIGN KEY (idD_SUBFUNCAO) 
        REFERENCES D_SUBFUNCAO(idD_SUBFUNCAO),
    CONSTRAINT fk_fdot_acao 
        FOREIGN KEY (idD_ACAO) 
        REFERENCES D_ACAO(idD_ACAO),
    CONSTRAINT fk_fdot_subacao 
        FOREIGN KEY (idD_SUBACAO) 
        REFERENCES D_SUBACAO(idD_SUBACAO),
    CONSTRAINT fk_fdot_eixo 
        FOREIGN KEY (idD_EIXO) 
        REFERENCES D_EIXO(idD_EIXO),
    CONSTRAINT fk_fdot_subeixo 
        FOREIGN KEY (idD_SUBEIXO) 
        REFERENCES D_SUBEIXO(idD_SUBEIXO),
    CONSTRAINT fk_fdot_tempo 
        FOREIGN KEY (idT_TEMPO) 
        REFERENCES D_TEMPO(idT_TEMPO),
    
    CONSTRAINT chk_valor_dotacao 
        CHECK (VALOR_DOTACAO >= 0),
    CONSTRAINT chk_status_dotacao 
        CHECK (STATUS IN ('ATIVA', 'BLOQUEADA', 'CANCELADA')),
    CONSTRAINT uq_codigo_dotacao_exercicio 
        UNIQUE (CODIGO_DOTACAO, EXERCICIO)
);

-- Indices para performance
CREATE INDEX idx_fdot_codigo 
    ON F_DOTACAO_ORCAMENTARIA(CODIGO_DOTACAO);
CREATE INDEX idx_fdot_exercicio 
    ON F_DOTACAO_ORCAMENTARIA(EXERCICIO);
CREATE INDEX idx_fdot_programa 
    ON F_DOTACAO_ORCAMENTARIA(idD_PROGRAMAS_PPAG);
CREATE INDEX idx_fdot_uo 
    ON F_DOTACAO_ORCAMENTARIA(idD_UO);
CREATE INDEX idx_fdot_funcao 
    ON F_DOTACAO_ORCAMENTARIA(idD_FUNCAO);
CREATE INDEX idx_fdot_acao 
    ON F_DOTACAO_ORCAMENTARIA(idD_ACAO);
CREATE INDEX idx_fdot_tempo 
    ON F_DOTACAO_ORCAMENTARIA(idT_TEMPO);
CREATE INDEX idx_fdot_status 
    ON F_DOTACAO_ORCAMENTARIA(STATUS);
CREATE INDEX idx_fdot_composto 
    ON F_DOTACAO_ORCAMENTARIA(EXERCICIO, STATUS, idD_PROGRAMAS_PPAG);
\end{lstlisting}

\section{Tabela Associativa - Classificação Temática}

\subsection{F\_DOTACAO\_ORCAMENTO\_TEMATICO}

\begin{lstlisting}
-- =====================================================
-- Tabela: F_DOTACAO_ORCAMENTO_TEMATICO
-- Relacionamento N:N entre dotacoes e orcamentos tematicos
-- Implementa classificacao com indices padronizados (20,50,85,100)
-- =====================================================
CREATE TABLE F_DOTACAO_ORCAMENTO_TEMATICO (
    idF_DOTACAO_ORCAMENTO_TEMATICO INT PRIMARY KEY AUTO_INCREMENT,
    
    -- Chaves Estrangeiras
    idF_DOTACAO_ORCAMENTARIA INT NOT NULL,
    idD_ORCAMENTO_TEMATICO INT NOT NULL,
    idD_INDICE_ALOCACAO INT NOT NULL,
    
    -- Metricas
    VALOR_ALOCADO DECIMAL(15,2) NOT NULL,
    
    -- Metadados
    EXERCICIO VARCHAR(4) NOT NULL,
    JUSTIFICATIVA_ALOCACAO TEXT NOT NULL,
    CRITERIO_ALOCACAO VARCHAR(200),
    META_FISICA VARCHAR(200),
    PUBLICO_ALVO_ESPECIFICO VARCHAR(200),
    
    -- Controle de Auditoria
    DATA_INCLUSAO DATETIME DEFAULT CURRENT_TIMESTAMP,
    USUARIO_INCLUSAO VARCHAR(50) NOT NULL,
    DATA_ALTERACAO DATETIME,
    USUARIO_ALTERACAO VARCHAR(50),
    STATUS VARCHAR(20) DEFAULT 'ATIVO',
    
    -- Foreign Keys
    CONSTRAINT fk_fdot_tematico_dotacao 
        FOREIGN KEY (idF_DOTACAO_ORCAMENTARIA) 
        REFERENCES F_DOTACAO_ORCAMENTARIA(idF_DOTACAO_ORCAMENTARIA)
        ON DELETE RESTRICT
        ON UPDATE CASCADE,
        
    CONSTRAINT fk_fdot_tematico_orcamento 
        FOREIGN KEY (idD_ORCAMENTO_TEMATICO) 
        REFERENCES D_ORCAMENTO_TEMATICO(idD_ORCAMENTO_TEMATICO)
        ON DELETE RESTRICT
        ON UPDATE CASCADE,
        
    CONSTRAINT fk_fdot_tematico_indice 
        FOREIGN KEY (idD_INDICE_ALOCACAO) 
        REFERENCES D_INDICE_ALOCACAO(idD_INDICE_ALOCACAO)
        ON DELETE RESTRICT
        ON UPDATE CASCADE,
    
    -- Unique Constraint - Uma dotacao nao pode ter mais de uma 
    -- classificacao no mesmo orcamento tematico no mesmo exercicio
    CONSTRAINT uq_dotacao_orcamento_exercicio 
        UNIQUE (idF_DOTACAO_ORCAMENTARIA, 
                idD_ORCAMENTO_TEMATICO, 
                EXERCICIO),
    
    -- Check Constraints
    CONSTRAINT chk_valor_alocado_tematico 
        CHECK (VALOR_ALOCADO >= 0),
    CONSTRAINT chk_status_tematico 
        CHECK (STATUS IN ('ATIVO', 'INATIVO', 'CANCELADO')),
    CONSTRAINT chk_justificativa_min 
        CHECK (LENGTH(JUSTIFICATIVA_ALOCACAO) >= 20),
    CONSTRAINT chk_exercicio_tematico 
        CHECK (LENGTH(EXERCICIO) = 4)
);

-- Indices para Performance
CREATE INDEX idx_fdot_tema_dotacao 
    ON F_DOTACAO_ORCAMENTO_TEMATICO(idF_DOTACAO_ORCAMENTARIA);
    
CREATE INDEX idx_fdot_tema_orcamento 
    ON F_DOTACAO_ORCAMENTO_TEMATICO(idD_ORCAMENTO_TEMATICO);
    
CREATE INDEX idx_fdot_tema_indice 
    ON F_DOTACAO_ORCAMENTO_TEMATICO(idD_INDICE_ALOCACAO);
    
CREATE INDEX idx_fdot_tema_exercicio 
    ON F_DOTACAO_ORCAMENTO_TEMATICO(EXERCICIO);
    
CREATE INDEX idx_fdot_tema_status 
    ON F_DOTACAO_ORCAMENTO_TEMATICO(STATUS);
    
CREATE INDEX idx_fdot_tema_composto 
    ON F_DOTACAO_ORCAMENTO_TEMATICO(
        idF_DOTACAO_ORCAMENTARIA, 
        EXERCICIO, 
        STATUS
    );

CREATE INDEX idx_fdot_tema_orcamento_exercicio 
    ON F_DOTACAO_ORCAMENTO_TEMATICO(
        idD_ORCAMENTO_TEMATICO, 
        EXERCICIO, 
        STATUS
    );
\end{lstlisting}

\section{Tabelas de Log e Auditoria}

\subsection{LOG\_ERROS\_ETL}

\begin{lstlisting}
-- =====================================================
-- Tabela: LOG_ERROS_ETL
-- Registro de erros durante processo ETL
-- =====================================================
CREATE TABLE LOG_ERROS_ETL (
    idLOG_ERRO INT PRIMARY KEY AUTO_INCREMENT,
    TABELA_DESTINO VARCHAR(100),
    PIPELINE VARCHAR(100),
    TIPO_ERRO VARCHAR(50),
    DESCRICAO_ERRO TEXT,
    REGISTRO_ERRO TEXT,
    DATA_ERRO DATETIME DEFAULT CURRENT_TIMESTAMP,
    USUARIO VARCHAR(50),
    EXERCICIO VARCHAR(4),
    STATUS_TRATAMENTO VARCHAR(20) DEFAULT 'PENDENTE',
    
    CONSTRAINT chk_status_tratamento 
        CHECK (STATUS_TRATAMENTO IN ('PENDENTE', 'TRATADO', 'IGNORADO'))
);

CREATE INDEX idx_log_tabela 
    ON LOG_ERROS_ETL(TABELA_DESTINO);
CREATE INDEX idx_log_data 
    ON LOG_ERROS_ETL(DATA_ERRO);
CREATE INDEX idx_log_status 
    ON LOG_ERROS_ETL(STATUS_TRATAMENTO);
\end{lstlisting}

\subsection{LOG\_EXECUCAO\_ETL}

\begin{lstlisting}
-- =====================================================
-- Tabela: LOG_EXECUCAO_ETL
-- Registro de execucoes dos pipelines ETL
-- =====================================================
CREATE TABLE LOG_EXECUCAO_ETL (
    idLOG_EXECUCAO INT PRIMARY KEY AUTO_INCREMENT,
    PIPELINE VARCHAR(100) NOT NULL,
    TABELA_DESTINO VARCHAR(100),
    DATA_INICIO DATETIME NOT NULL,
    DATA_FIM DATETIME,
    STATUS_EXECUCAO VARCHAR(20),
    REGISTROS_PROCESSADOS INT,
    REGISTROS_INSERIDOS INT,
    REGISTROS_ATUALIZADOS INT,
    REGISTROS_ERRO INT,
    TEMPO_EXECUCAO_SEGUNDOS INT,
    MENSAGEM TEXT,
    USUARIO VARCHAR(50),
    EXERCICIO VARCHAR(4),
    
    CONSTRAINT chk_status_execucao 
        CHECK (STATUS_EXECUCAO IN ('EXECUTANDO', 'SUCESSO', 'ERRO', 'CANCELADO'))
);

CREATE INDEX idx_log_exec_pipeline 
    ON LOG_EXECUCAO_ETL(PIPELINE);
CREATE INDEX idx_log_exec_data 
    ON LOG_EXECUCAO_ETL(DATA_INICIO);
CREATE INDEX idx_log_exec_status 
    ON LOG_EXECUCAO_ETL(STATUS_EXECUCAO);
\end{lstlisting}

\section{Views de Apoio}

\subsection{VW\_VALIDACAO\_INDICES}

\begin{lstlisting}
-- =====================================================
-- View: VW_VALIDACAO_INDICES
-- Validacao da soma de indices por dotacao
-- =====================================================
CREATE VIEW VW_VALIDACAO_INDICES AS
SELECT 
    fdo.idF_DOTACAO_ORCAMENTARIA,
    fdo.CODIGO_DOTACAO,
    fdo.NOME_DOTACAO,
    fdo.VALOR_DOTACAO,
    fdot.EXERCICIO,
    COUNT(DISTINCT fdot.idD_ORCAMENTO_TEMATICO) as QTD_ORCAMENTOS,
    SUM(ia.VALOR_INDICE) as SOMA_INDICES,
    SUM(fdot.VALOR_ALOCADO) as SOMA_VALORES_ALOCADOS,
    STRING_AGG(
        CONCAT(dot.SIGLA, '(', ia.VALOR_INDICE, ')'), 
        ' + ' ORDER BY ia.VALOR_INDICE DESC
    ) as DETALHE_CLASSIFICACOES,
    CASE 
        WHEN SUM(ia.VALOR_INDICE) = 100 THEN 'OK_COMPLETO'
        WHEN SUM(ia.VALOR_INDICE) < 100 THEN 'OK_PARCIAL'
        WHEN SUM(ia.VALOR_INDICE) <= 200 THEN 'ATENCAO'
        ELSE 'CRITICO'
    END as STATUS_VALIDACAO
FROM F_DOTACAO_ORCAMENTARIA fdo
JOIN F_DOTACAO_ORCAMENTO_TEMATICO fdot
    ON fdo.idF_DOTACAO_ORCAMENTARIA = fdot.idF_DOTACAO_ORCAMENTARIA
JOIN D_INDICE_ALOCACAO ia
    ON fdot.idD_INDICE_ALOCACAO = ia.idD_INDICE_ALOCACAO
JOIN D_ORCAMENTO_TEMATICO dot
    ON fdot.idD_ORCAMENTO_TEMATICO = dot.idD_ORCAMENTO_TEMATICO
WHERE fdot.STATUS = 'ATIVO'
  AND fdo.STATUS = 'ATIVA'
GROUP BY 
    fdo.idF_DOTACAO_ORCAMENTARIA,
    fdo.CODIGO_DOTACAO,
    fdo.NOME_DOTACAO,
    fdo.VALOR_DOTACAO,
    fdot.EXERCICIO;
\end{lstlisting}

\subsection{VW\_RESUMO\_ORCAMENTOS\_TEMATICOS}

\begin{lstlisting}
-- =====================================================
-- View: VW_RESUMO_ORCAMENTOS_TEMATICOS
-- Consolidacao por orcamento tematico
-- =====================================================
CREATE VIEW VW_RESUMO_ORCAMENTOS_TEMATICOS AS
SELECT 
    fdot.EXERCICIO,
    dot.CODIGO_ORCAMENTO,
    dot.NOME_ORCAMENTO,
    dot.SIGLA,
    COUNT(DISTINCT fdot.idF_DOTACAO_ORCAMENTARIA) as QTD_DOTACOES,
    SUM(fdot.VALOR_ALOCADO) as VALOR_TOTAL,
    AVG(fdot.VALOR_ALOCADO) as VALOR_MEDIO,
    MIN(fdot.VALOR_ALOCADO) as VALOR_MINIMO,
    MAX(fdot.VALOR_ALOCADO) as VALOR_MAXIMO,
    COUNT(CASE WHEN ia.VALOR_INDICE = 100 THEN 1 END) as QTD_EXCLUSIVAS,
    COUNT(CASE WHEN ia.VALOR_INDICE = 85 THEN 1 END) as QTD_PREDOMINANTES,
    COUNT(CASE WHEN ia.VALOR_INDICE = 50 THEN 1 END) as QTD_MODERADAS,
    COUNT(CASE WHEN ia.VALOR_INDICE = 20 THEN 1 END) as QTD_TANGENCIAIS
FROM F_DOTACAO_ORCAMENTO_TEMATICO fdot
JOIN D_ORCAMENTO_TEMATICO dot
    ON fdot.idD_ORCAMENTO_TEMATICO = dot.idD_ORCAMENTO_TEMATICO
JOIN D_INDICE_ALOCACAO ia
    ON fdot.idD_INDICE_ALOCACAO = ia.idD_INDICE_ALOCACAO
WHERE fdot.STATUS = 'ATIVO'
GROUP BY 
    fdot.EXERCICIO,
    dot.CODIGO_ORCAMENTO,
    dot.NOME_ORCAMENTO,
    dot.SIGLA;
\end{lstlisting}

\section{Triggers}

\subsection{Trigger: Calcular Valor Alocado Automaticamente}

\begin{lstlisting}
-- =====================================================
-- Trigger: trg_calcula_valor_alocado
-- Calcula VALOR_ALOCADO baseado no indice e valor da dotacao
-- =====================================================
DELIMITER $$

CREATE TRIGGER trg_calcula_valor_alocado
BEFORE INSERT ON F_DOTACAO_ORCAMENTO_TEMATICO
FOR EACH ROW
BEGIN
    DECLARE v_valor_dotacao DECIMAL(15,2);
    DECLARE v_indice INT;
    
    -- Buscar valor da dotacao
    SELECT VALOR_DOTACAO INTO v_valor_dotacao
    FROM F_DOTACAO_ORCAMENTARIA
    WHERE idF_DOTACAO_ORCAMENTARIA = NEW.idF_DOTACAO_ORCAMENTARIA;
    
    -- Buscar valor do indice
    SELECT VALOR_INDICE INTO v_indice
    FROM D_INDICE_ALOCACAO
    WHERE idD_INDICE_ALOCACAO = NEW.idD_INDICE_ALOCACAO;
    
    -- Calcular se nao foi informado ou esta zerado
    IF NEW.VALOR_ALOCADO IS NULL OR NEW.VALOR_ALOCADO = 0 THEN
        SET NEW.VALOR_ALOCADO = v_valor_dotacao * (v_indice / 100.0);
    END IF;
END$$

DELIMITER ;
\end{lstlisting}

\subsection{Trigger: Atualizar Data de Alteração}

\begin{lstlisting}
-- =====================================================
-- Trigger: trg_atualiza_data_alteracao
-- Atualiza DATA_ALTERACAO automaticamente
-- =====================================================
DELIMITER $$

CREATE TRIGGER trg_atualiza_data_alteracao
BEFORE UPDATE ON F_DOTACAO_ORCAMENTO_TEMATICO
FOR EACH ROW
BEGIN
    SET NEW.DATA_ALTERACAO = NOW();
END$$

DELIMITER ;
\end{lstlisting}

\section{Stored Procedures}

\subsection{Procedure: Validar Soma de Índices}

\begin{lstlisting}
-- =====================================================
-- Procedure: sp_validar_soma_indices
-- Valida se soma de indices esta consistente
-- =====================================================
DELIMITER $$

CREATE PROCEDURE sp_validar_soma_indices(
    IN p_exercicio VARCHAR(4),
    IN p_limite_critico INT
)
BEGIN
    SELECT 
        fdo.CODIGO_DOTACAO,
        fdo.NOME_DOTACAO,
        fdo.VALOR_DOTACAO,
        SUM(ia.VALOR_INDICE) as SOMA_INDICES,
        COUNT(*) as QTD_CLASSIFICACOES,
        STRING_AGG(dot.SIGLA, ', ') as ORCAMENTOS,
        CASE 
            WHEN SUM(ia.VALOR_INDICE) <= 100 THEN 'OK'
            WHEN SUM(ia.VALOR_INDICE) <= 200 THEN 'ATENCAO'
            ELSE 'CRITICO'
        END as STATUS
    FROM F_DOTACAO_ORCAMENTARIA fdo
    JOIN F_DOTACAO_ORCAMENTO_TEMATICO fdot
        ON fdo.idF_DOTACAO_ORCAMENTARIA = fdot.idF_DOTACAO_ORCAMENTARIA
    JOIN D_INDICE_ALOCACAO ia
        ON fdot.idD_INDICE_ALOCACAO = ia.idD_INDICE_ALOCACAO
    JOIN D_ORCAMENTO_TEMATICO dot
        ON fdot.idD_ORCAMENTO_TEMATICO = dot.idD_ORCAMENTO_TEMATICO
    WHERE fdot.EXERCICIO = p_exercicio
      AND fdot.STATUS = 'ATIVO'
      AND fdo.STATUS = 'ATIVA'
    GROUP BY 
        fdo.CODIGO_DOTACAO,
        fdo.NOME_DOTACAO,
        fdo.VALOR_DOTACAO
    HAVING SUM(ia.VALOR_INDICE) > p_limite_critico
    ORDER BY SUM(ia.VALOR_INDICE) DESC;
END$$

DELIMITER ;
\end{lstlisting}

\end{document}